\documentclass[english,10pt,letterpaper,onecolumn]{IEEEtran} 
  \usepackage[margin=.75in]{geometry}
  \usepackage{graphicx}
  \usepackage[utf8]{inputenc} 
  \usepackage[noadjust]{cite}
  \usepackage{babel}  
  \usepackage{titling}
  \usepackage{listings}
  \usepackage{url}
  \usepackage[usenames,dvipsnames,svgnames,table]{xcolor}
  \usepackage{color}
  \lstset{%
    basicstyle=\ttfamily\footnotesize,
    backgroundcolor=\color{gray!10},%
    tabsize=3,
    language=bash,
    columns=fullflexible,
    %frame=shadowbox,
    frame=leftline,
    rulesepcolor=\color{blue},
    xleftmargin=20pt,
    framexleftmargin=15pt,
    keywordstyle=\color{black}\bf,
    commentstyle=\color{OliveGreen},
    %stringstyle=\color{red},
    breaklines=true,
    %numbers=left,
    %numberstyle=\tiny,
    %numbersep=5pt,
    showstringspaces=false,
    postbreak=\mbox{{/}\space},
    emph={str},emphstyle={\color{magenta}},
    alsoletter=-,%to allow qemu-system...
    %add extra keywords here.
    emph={%  
      cd, cp, qemu-system-i386, mv, mkdir, make, for, git, gdb, uname, tar, chmod, vim%
      },emphstyle={\color{black}\bf},
    %remove keywords below
    deletekeywords={enable, continue}
  }%
  
  % TITLE
  \title{Project 1: Getting Acquainted}
  \author{
    Deval Kaku \hspace{.5cm}
    \and
    Jing Huang \hspace{.5cm}
    \and
    Saeed Khorram
  }
  \date{April 15th, 2018}
  
  \begin{document}
  \begin{titlepage} 
  \pagenumbering{gobble}
  \maketitle
  \begin{center}
  CS544\\
  Operating Systems II\\
  Spring 2018
  \vspace{50 mm}
  \end{center}
  
  % ABSTRACT
  \begin{abstract}
  In this assignment, we run the commands which are used to setup the Yocto version of Linux which is running on a Qemu virtual machine. This document also contains a command log used to setup the virtual machine and an explanation of flags used in the process 
  \end{abstract}
  \end{titlepage}
  
  \IEEEpeerreviewmaketitle
  \pagenumbering{arabic}
  \section{Log of Commands}
  \flushleft
  Create the group folder and change the permission.
  \begin{lstlisting}
  cd /scratch/spring2018
  mkdir 12
  chmod -R 777 12
  \end{lstlisting}
  Clone the kernel image to the group folder.
  \begin{lstlisting}
  cd 12
  git clone git://git.yoctoproject.org/linux-yocto-3.19
  \end{lstlisting}
  Switch to the “v3.19.2” tag.
  \begin{lstlisting}
  git checkout tags/v3.19.2
  make tags
  \end{lstlisting}
  Copy all the necessary files to the group folder, source the environment variables, and then build the kernel.
  \begin{lstlisting}
  cp /scratch/files/* /scratch/spring2018/12/
  cp config-3.19.2-yocto-standard ./linux-yocto-3.19/.config
  source environment-setup-i586-poky-linux
  make -j4 all
  \end{lstlisting}
  Run QEMU emulator.
  \begin{lstlisting}
  qemu-system-i386 -gdb tcp::5512 -S -nographic -kernel bzImage-qemux86.bin -drive file=core-image-lsb-sdk-qemux86.ext4,if=virtio -enable-kvm -net none -usb -localtime -- no-reboot --append "root=/dev/vda rw console=ttyS0 debug"
  \end{lstlisting}
  Open a new terminal (the second one), and run gdb debugger.
  \begin{lstlisting}
  cd /scratch/spring2018/12/linux-yocto-3.19
  gdb vmlinux
  \end{lstlisting}
  Boot the built kernel.
  \begin{lstlisting}
  (gdb) target remote: 5512
  (gdb) continue
  \end{lstlisting}
  
  \newpage
  \section{Explanation Of Flags}
  
  \begin{itemize}
  \item -gdb tcp::5512 \\
  Open a GDB server on TCP port 5512.
  \item -S \\
  Do not start CPU at startup.
  \item -nographic \\
  Disable graphical output so QEMU is just a command line application.
  \item -kernel bzImage-qemux86.bin \\
  Use bzImage-qemux86.bin as kernel image for the VM.
  \item -drive file=core-image-lsb-sdk-qemux86.ext3,if=virtio \\
  Define a new drive with disk image as core-image-lsb-sdk-qemux86.ext4. The ‘if’ option defines on which type on interface the drive is connected. In this case it is virtio, which is short for virtualization.
  \item -enable-kvm \\
  Enables KVM full virtualization support.
  \item -net none \\
  This setting is for network devices. The option "none" specifies to not configure any network devices.
  \item -usb \\ 
  This enables the USB driver.
  \item -localtime \\ 
  Set the VM to sync with local time.
  \item --no-reboot \\
  Exit instead of rebooting.
  \item --append "root=/dev/vda rw console=ttyS0 debug" \\
  Use "root=/dev/vda rw console=ttyS0 debug" as kernel command line. Specifically, “root=/dev/vda” specifies where the root directory is to mount and boot correctly. “console=ttyS0” tells the kernel to use the “ttyS0” console.
  \end{itemize}
  
  \newpage
  \section{Version Control Log}
  
  \begin{center}
   \begin{tabular}{||c c c ||} 
   \hline
   Date & Author & Commit \\ 
   \hline\hline
   4/15/18 & Jing Huang & 752ca01c049820caee0c25eee2eb6699199983eb\\ 
   \hline
   4/15/18 & Jing Huang & cf2c5abbd53a5e81c70c30d7fd618ef291878a3f\\
   \hline
  \end{tabular}
  \end{center}
  
  \section{Work Log}
  
  \begin{center}
   \begin{tabular}{||c c c ||} 
   \hline
   Date & Hours & Work Done\\ 
   \hline\hline
   4/13/18 & 1 & Group meeting, cloned repo and setup the VM\\ 
   \hline
    4/14/18 & 2 & Group meeting, started writeup\\ 
   \hline
   4/15/18 & 4 & Finished writeup, made makefile, submitted\\
   [1ex] 
   \hline
  \end{tabular}
  \end{center}
  
  % references, IEEE format
  %\subsection*{\bf References}
  %\begin{enumerate}
  %\item S.Weil, "QEMU Binaries for Windows and QEMU Documentation" Dec 21, 2017 {\textit {QEMU Emulator User Documentation}} [Online] Available: https://qemu.weilnetz.de/w64/2012/2012-12-04/qemu-doc.html [Accessed on Apr 07, 2018]
  %\end{enumerate}
  
  \end{document}